\documentclass[autodetect-engine,dvipdfmx-if-dvi,ja=standard,a4paper,layout=v2]{bxjsreport}
\usepackage{ifxetex,ifluatex}
\usepackage{unicode-math}
\setmainfont[Ligatures=TeX]{TeXGyreTermes}
\setsansfont[Ligatures=TeX]{TeXGyreHeros}
\setmonofont{Inconsolatazi4}
\setmathfont{TeXGyreTermesMath}
\usepackage{graphicx}
\author{前田大輝}
\title{周期ポテンシャルにおける波束の時間発展}
\begin{document}
    \maketitle
    \begin{abstract}
      周期ポテンシャルは一定の周期構造を持つ分子の並びをモデル化したものであり、
      中でも最も簡単なKronig-Pennyモデルでは定常状態のエネルギーバンド構造を説明することができる。
      一方で時間に依存した現象では、波束が時間の2乗よりも早く緩和することが知られており、
      heyperdiffusionと呼ばれている。
      このような、周期ポテンシャル中の粒子が真空中とは違った振る舞いをすることに関する知見は、
      固体物理やナノリソグラフィなどの分野に応用できる場合がある。
      そこで本研究は、特定の(半)周期ポテンシャルにおける波束が
      どのように緩和していくのかを解析することを目的として、
      主にコンピュータにおける数値計算的な手法を用いている。
      その結果Kronig-Penny型の周期ポテンシャルに入射する波束は、
      ステップ型のポテンシャルに入射した場合と比べて、
      反射波に遅れが生じることを発見した。
    \end{abstract}
    \tableofcontents
    \chapter{研究の目的}
    \begin{chapterabstract}
      物質中の電子は真空中と比べ、特異な振る舞いをすることが知られている。
      その代表例がエネルギーバンド構造であり、物質中の電子は連続したエネルギースペクトルを持つことができない。
      バンドギャップにあたるエネルギー状態をとる電子は物質境界で全反射するか、ギャップ分のエネルギーを光子として放出し、
      バンドに適合したエネルギーを持って物質中に侵入する。\par
      この物理現象について、定常状態での解析はなされているが、時間依存の詳細についてはまだ未解明である。
      近年は光格子時計により時間分解能が飛躍的に向上しているため、現在の「一瞬にしてバンド構造が作られる」という近似が
      成立しなくなる可能性がある。そこで、どの程度の時間分解能ならばその近似が成立するかの目安を求めること重要性が増している。\par
    \end{chapterabstract}
    \section{周期ポテンシャル}
    周期ポテンシャルは物質中に原子が周期的に並んでいる状況をモデル化したものであり、
    その最も簡単な例が箱型ポテンシャルを周期的に並べたKronig-Pennyポテンシャルである。
    このモデルは簡単な構造ながらエネルギーバンド構造を説明することができるため、
    周期ポテンシャルは物質中の粒子の振る舞いを表すモデルとして優れていることの実例となっている。\par
    エネルギーバンドの構成時間を調べるために、この最も簡単なKronig-Pennyモデルを用いて解析を行う。
    \section{波束の時間発展}

    \chapter{先行研究}
    \chapter{アプローチ}
    \chapter{結果}
    \chapter{考察}
    \chapter{schrpyについて}
\end{document}
