\documentclass[autodetect-engine,dvipdfmx-if-dvi,ja=standard,a4paper,layout=v2]{bxjsreport}
\usepackage{ifxetex,ifluatex,unicode-math}
\setmainfont[Ligatures=TeX]{TeXGyreTermes}
\setsansfont[Ligatures=TeX]{TeXGyreHeros}
\setmonofont{Inconsolatazi4}
\usepackage{graphicx,braket, amsmath, ascmac}
\geometry{top= 10truemm, bottom= 5 truemm, left= 25truemm, right= 25truemm}
\makeatletter
\newcommand{\Tref}[1]{Tab.~\ref{#1}}
\newcommand{\Eref}[1]{eq.~(\ref{#1})}
\newcommand{\Fref}[1]{fig.~\ref{#1}}
\renewcommand{\theequation}{
    \thesection.\arabic{equation}}
\@addtoreset{equation}{chapter}
\def\theequation{\thechapter.\arabic{equation}}
\def\@cite#1{\textsuperscript{#1)}}
\makeatother

\author{前田大輝}
\title{周期ポテンシャルにおける波束の時間発展}
\begin{document}
    \maketitle
    \begin{abstract}
        周期ポテンシャルは一定の周期構造を持つ分子の並びをモデル化したものであり、
        中でも最も簡単なKronig-Pennyモデルでは定常状態のエネルギーバンド構造を説明することができる。
        一方で時間に依存した現象では、波束が時間の2乗よりも早く緩和することが知られており、
        heyperdiffusionと呼ばれている。
        このような、周期ポテンシャル中の粒子が真空中とは違った振る舞いをすることに関する知見は、
        固体物理やナノリソグラフィなどの分野に応用できる場合がある。
        そこで本研究は、特定の(半)周期ポテンシャルにおける波束が
        どのように緩和していくのかを解析することを目的として、
        主にコンピュータにおける数値計算的な手法を用いている。
        その結果Kronig-Penny型の周期ポテンシャルに入射する波束は、
        ステップ型のポテンシャルに入射した場合と比べて、
        反射波に遅れが生じることを発見した。
    \end{abstract}
    \tableofcontents
    \chapter{研究の目的}
    \begin{chapterabstract}
        物質中の電子は真空中と比べ、特異な振る舞いをすることが知られている。
        その代表例が結晶内におけるエネルギーバンド構造であり、結晶中の電子は連続したエネルギースペクトルを持つことができない。
        バンドギャップにあたるエネルギー状態をとる電子は結晶境界で全反射するか、ギャップ分のエネルギーを光子として放出し、
        バンドに適合したエネルギーを持って物質中に侵入する。\par
        この物理現象について、定常状態での解析はなされているが、時間依存の詳細についてはまだ未解明である。
        近年は光格子時計により時間分解能が飛躍的に向上しているため、現在の「一瞬にしてバンド構造が作られる」という近似が
        成立しなくなる可能性がある。そこで、どの程度の時間分解能ならばその近似が成立するかの目安を求めること重要性が増している。\par
    \end{chapterabstract}
    \section{Kronig-Pennyモデル}
    周期ポテンシャルは物質中に原子が周期的に並んでいる状況をモデル化したものであり、
    その最も簡単な例が箱型ポテンシャルを周期的に並べたKronig-Pennyポテンシャルである。
    このモデルは簡単な構造ながらエネルギーバンド構造を説明することができるため、
    周期ポテンシャルの有用性を示す実例となっている。\par
    よって、エネルギーバンドの構成時間を調べるために、この最も簡単なKronig-Pennyモデルを用いて解析を行う。
    またKronig-Pennyモデル自体がナノリソグラフィーの素子や半導体超格子に対する良い近似として成立しているため、
    それらの性質を知る上で、この研究は実際的な研究となる。
    \section{波束の時間発展}
    Schrödinger方程式におけるGauss型波束は時間発展とともに緩和していく。
    その速度の目安として分散$\sigma^2$の時間依存性を使うことができる。
    \section{数値計算}
    Kronig-Pennyポテンシャル中のGauss型波束の時間発展を解析的に得ることは非常に困難を伴う。
    そのため、近似的な方法を用いる必要がある。
    一般にSchrődinger方程式の近似的解法には摂動法、くりこみ法、数値解析法が存在するため、それらを用いて解析を行う。\par
    中でも数値解析法は特別な条件を系に課す必要がないため、主要な解析法として用いることができる。
    数値的にSchrődinger方程式を解く方法には大きくFFT法と有限要素法が存在する。
    \subsection{FFT法}
    FFT法はハミルトニアンの固有状態を重ね合わせることによって系の初期状態を構成し、時間発展を追う方法である。
    \begin{align}
        \psi(x,t)=\sum_{\mathrm{n}}^{\infty}C_{\mathrm{n}}\ \phi_{n}(x)\ \mathrm{e}^{-\frac{i}{\hbar}\varepsilon_n t}
    \end{align}
    適当な$C_{n}$を与えれば、その後の時間発展を位相をずらすだけで表現できる。
    この方法では固有状態がわかってしまえば比較的短い計算でも高い精度で計算することができる。
    誤差は主にフーリエ変換の連続的な積分を離散的な和で表すことによって発生する。
    この相対誤差は低周波領域で顕著なため、長時間の時間発展を計算する際は誤差が大きくなるが、
    短時間の時間発展ではかなり効率がいい。
    また、あらゆる時刻の時間発展を計算する場合でも定数時間で計算が終了するという点でも非常に性質が良い。\par
    ただし、Kronig-Pennyポテンシャルにおける定常状態のSchrődinger方程式は、Diracの櫛と呼ばれる極限を除いて、
    解析的に解を得ることに計算精度の面で困難がある。
    そこで、本研究では解析的に定常状態を得ることは避け、数値的に定常状態を得るだけに留めることにする。
    数値的に得られる定常状態であっても、誤差の範囲内では真の定常状態と一致する。
    \subsection{有限要素法}
    もうひとつの数値解法である有限要素法は積分する範囲を有限の区間に分割し、
    分割した区間内での解は適当な可積分関数(冪関数など)で表せると近似することで、
    微分方程式の近似解を得る方法である。\par
    この方法はバラエティが豊かだが、基本的に安定域と計算量の2つが重要な指標となる。
    計算量には、空間的計算資源(メモリ消費量)と時間的計算資源(計算時間)の消費量という2つの意味を持つが、
    現代の計算機の性能で空間的計算量が問題になることは基本的に無いため、計算時間のことを指して計算量という言葉を用いる。
    安定域とは計算誤差が蓄積しない誤差の領域で、多くの計算方法は安定域内で真の値の周辺を振動するように計算誤差が発生する。
    逆に、安定域の外側では誤差が蓄積し、計算を進めると全く信用できない値が出てくるようになる。
    基本的に計算量が少なく、安定域が広い計算方法が求められるが、これらはある程度トレードオフの関係にある。\par
    話を有限要素法に戻すと、この方法のバリエーションとして高階法と多段法が存在する。
    高階法は積分したい関数を一次関数で逐次近似し、高次の影響について補正を加える方法で、Runge-Kutta法やEulor法などが知られている。
    %オイラー法の式
    上式はEulor法の展開公式で、$t=t_n$の値がわかれば、$t=t_{n+1}$の値が計算できることがわかる。
    Runge-Kutta法はこの方法に補正を加えたものになる。\par
    一方、多段法は積分したい関数をRagrange補間などによってべき級数に近似する方法で、
    予測子修正子Adams法やAdams-Bashforth法などが知られている。
    %Adams-Bashforth法の式
    上式はAdams-Bashfoeth法の展開公式で、$t=t_n,t_{n-1}...t_{n-i}...t_{n-N}$の$N$個の値から$t_{n+1}$の値を計算することができる。
    予測子修正子Adams法はAdams-Bashforth法に補正を加えたもので、過去の値を必要とすることはおなじである。\par
    一般に$N$を増やせば精度を上げることができるが、その精度は残念ながら$N$には比例しない。
    経験的に予測子修正子Adams法で7次の精度の計算をするためには、8個の過去の値が必要になることが知られている。\par
    高階法は計算が単純なため、高次の項が大きくないような性質の良い関数では多段法に比べて高速に終わる。
    しかし、激しく振動する関数ではTaylor級数の高次の項の影響をもろに受けるため、安定域の外に出やすい。
    Kronig-Pennyモデルでの計算でステップ数を増やして誤差を安定域に留めようとすると、結局計算時間が長くなってしまうことがある。
    一方、多段法は補間を行う際に高次の振動を取り去り、なめらかにするため、安定域が広い。
    ある程度の誤差を許容すれば高階法よりも比較的早く計算が終わる。
    Kronig-Pennyモデルでの波束はポテンシャルによって複雑に散乱されるため、高次の振動が多く起こる。
    そこで、安定域の広い予測子修正子Adams法を用いることにする。
    \chapter{先行研究}
    \begin{chapterabstract}
        Kronig-Pennyモデルは、Blochの定理を適用可能な実例として提案された。
        箱型ポテンシャルが周期的に並んでいる状況を設定したモデルで、
        KronigとPennyはこのモデルの固有状態を計算することによって、
        周期ポテンシャルにエネルギーバンド構造が現れることを説明した。
        Kronig-Pennyモデル中の波束の時間発展については、Zhenjunによる数値計算が行われている。
        この中で、重心が止まった波束の緩和速度について、分散の時間発展を使うことで数値化が行われた。
    \end{chapterabstract}
    \section{Blochの定理とKronig-Pennyモデルの定常解}
    ハミルトニアンに並進対称性を与えると定常状態の波動関数が周期関数と三角関数の積で表せるというものがBlochの定理の内容である。
    %Blochの定理
    これを今から証明する。\par
    証明のために演算子の交換関係についての有用な定理を証明する。
    あるHermit演算子$A$、$B$が交換関係$[A,\ B]=0$を満たすとき$A$、$B$は同時固有状態を持つ。
    $A$の固有状態を$\ket{A}$固有値を$\alpha$とする。
    \section{Zenjunの超拡散}
    \chapter{アプローチ}
    \chapter{結果}
    \chapter{考察}
    \chapter{schrpyについて}
\end{document}
